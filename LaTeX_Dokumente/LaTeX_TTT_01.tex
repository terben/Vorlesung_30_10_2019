\documentclass[14pt, a4paper]{scrartcl}

\usepackage[ngerman]{babel}
\usepackage[utf8]{inputenc}
\usepackage[T1]{fontenc}
\usepackage{csquotes}

% 'besser aussehende' Tabellen - dies wird in einem anderem
% Video dieser Reihe vorgestellt werden.
\usepackage{tabularx}
\usepackage{booktabs}
\usepackage{hyperref}

\begin{document}
%
\section*{Von Minuszeichen, Binde- und Gedankenstrichen}
Man beachte die \emph{verschiedenen Strichsymbole} bei der Zahl $-9$, dem Gedankenstrich -- so wie hier --, dem Zahlenbereich 2--5 und dem Bindestrich wie in der Herrmann-Harry-Schmitz-Straße!

Unter anderem zur Beschreibung von Unix-Kommandos, wie \texttt{ls -{}-reverse}, braucht man auch hin und wieder einen \emph{doppelten Strich}.

Tabelle~\ref{tab:strichsymbole} fasst das Wesentliche zusammen. Auf der Seite
\href{https://de.wikipedia.org/wiki/Wikipedia:Typografie#Bindestrich.2C_Gedankenstrich.2C_Minuszeichen_und_andere_waagerechte_Striche}{Wikipedia:Typografie} finden Sie nähere Erläuterungen zu dem Thema.
%
\begin{table}[h]
  \centering
  \captionabove{Strichsymbole und wie sie in \LaTeX{} gesetzt werden. In der ersten Spalte ist zur Verdeutlichung des verschiedenen Aussehens jeweils ein \enquote{+} vorangestellt.}
  \label{tab:strichsymbole}
  \begin{tabularx}{\textwidth}{ccX}
    \toprule
	Strichsymbol & \LaTeX--Quelltext & Bedeutung und Erläuterungen \\
	\midrule
	$+-$ & \$-\$ & Minuszeichen. Bitte achten Sie darauf, im
	Fliesstext negative Zahlen korrekt zu setzen. Korrekt gesetzt (\$-9\$ in \LaTeX--Quelltext) sieht es so aus: $-9$. Falsch gesetzt (-9 im \LaTeX--Quelltext) sieht es so aus: -9.\\
	+- & - & Bindestrich; dient zur Trennung zusammengesetzter Worte. \\
	+-- & -{}- & Gedankenstrich. Neben dem Gedankenstrich dient dieses Symbol unter Anderem zur Darstellung von Zahlenbereichen wie in \enquote{1--5} oder zum Absetzen
	von Zahlen und Buchstaben wie in \enquote{D--53121}. \\
    +-{}- & -\{\}- & Um zwei Bindestriche direkt hintereinander zu setzen, braucht man dieses Konstrukt, da \enquote{-{}-} der Gendankenstrich ist. \\
	\bottomrule
  \end{tabularx}
\end{table}

\end{document}
