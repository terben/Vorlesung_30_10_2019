% Kommentierte Quelldatei zu Video 05 der LaTeX Tipps-Tricks-Techniken
% Serie von Thomas Erben
% (siehe )

\documentclass[12pt,a4paper]{scrartcl}

% Deutsche Spracheinstellungen sind unter anderem für eine korrekte
% Trennung nach deutschen Regeln verantwortlich (erster Satz unten)
\usepackage[ngerman]{babel}
\usepackage[utf8]{inputenc}
% Die T1-Schriften behandeln Worte mit deutschen Umlauten korrekt,
% u.a. Trennung von Worten mit Umlauten
\usepackage[T1]{fontenc}
\usepackage{csquotes}
% Mit dem folgenden Paket können Worte in Tele-type-Fonts getrennt werden
\usepackage[htt]{hyphenat}

\usepackage{parskip}

\begin{document}
%
\section{Probleme bei der Worttrennung}
Es gibt viele naturwissenschaftliche Fächer: Biologie, Physik,
Informatik, Chemie, Mathematik.

Es gibt viele Arten von Fahrzeugen: Rennwagen, Oldtimer, Traktoren,
Smarts, Geländewagen.

% Worte mit Bindestrichen werden standardmäßig von LaTeX nur an den
% Bindestrichen getrennt. Dies kann mit deutschen Spracheinstellungen
% mit dem Konstrukt '"=' für den Bindestrich vermieden werden.
Worte mit Bindestrichen: Tschebyscheff-Funktion,
Kfz-Unfallversicherung, Unix"=Kommandos, öffentlich-rechtlich.

% Trennungen von Worten in Schreibmaschinenschrift sollten im
% Einzelfall behandelt werden.
Laden Sie sich von \textit{eCampus} aus dem Ordner für die heutige
Übung die Datei \texttt{uebung\_01\_dateien.tgz} herunter.
\end{document}
